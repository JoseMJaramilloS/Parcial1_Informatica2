\documentclass{article}
\usepackage[utf8]{inputenc}
\usepackage[spanish]{babel}
\usepackage{listings}
\usepackage{graphicx}
\graphicspath{ {images/} }
\usepackage{cite}

\begin{document}

\begin{titlepage}
    \begin{center}
        \vspace*{1cm}
            
        \Huge
        \textbf{Desafio 1}
            
        \vspace{0.5cm}
        \LARGE
        Informa2 S.A.S
            
        \vspace{1.5cm}
            
        \textbf{Victor Manuel Jimenez Garcia\\
                Jose Miguel Jaramillo Sanchez\\
                Sebastian Garcia Morales}

        \vfill
            
        \vspace{0.8cm}
            
        \Large
        Despartamento de Ingeniería Electrónica y Telecomunicaciones\\
        Universidad de Antioquia\\
        Medellín\\
        Febrero 17 de 2022
            
    \end{center}
\end{titlepage}

\tableofcontents

\newpage
\section{Objetivos}\label{objetivos}
\section{Introduccion}\label{intro}
\section{Marco Teorico}\label{marco}

\subsection{Conocimientos previos}


\section{Analisis del problema} \label{analisis}
Esta sección es para agregar toda la información correspondiente con código, citas, etc.\newline
Entre los aspectos fundamentales, se debe tener en cuenta realizar la citación respectiva de la información.\newline
Se muestra como agregar una imagen a nuestro informe.


\section{Conclusiones} \label{conclusiones}


\bibliographystyle{IEEEtran}
\bibliography{references}

\end{document}
