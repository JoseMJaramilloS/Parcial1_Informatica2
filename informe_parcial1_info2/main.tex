\documentclass{article}
\usepackage[utf8]{inputenc}
\usepackage[spanish]{babel}
\usepackage{listings}
\usepackage{graphicx}
\graphicspath{ {images/} }
\usepackage{cite}

\begin{document}

\begin{titlepage}
    \begin{center}
        \vspace*{1cm}
            
        \Huge
        \textbf{Desafio 1}
            
        \vspace{0.5cm}
        \LARGE
        Informa2 S.A.S
            
        \vspace{1.5cm}
            
        \textbf{Victor Manuel Jimenez Garcia\\
                Jose Miguel Jaramillo Sanchez\\
                Sebastian Garcia Morales}

        \vfill
            
        \vspace{0.8cm}
            
        \Large
        Despartamento de Ingeniería Electrónica y Telecomunicaciones\\
        Universidad de Antioquia\\
        Medellín\\
        Febrero 17 de 2022
            
    \end{center}
\end{titlepage}

\tableofcontents

\newpage
\section{Objetivos}\label{objetivos}
\section{Introduccion}\label{intro}
\section{Marco Teorico}\label{marco}

\subsection{Conocimientos previos}

El sistema de encriptacion intercambia informacion por medio de la comunicacion serial, que es el proceso de enviar datos de caracter binario un bit a la vez.

Para desencriptar, es necesario paralelizar dicha secuencia de bits que luego seran las entradas de un circuito de logica combinacional encargado de comparar los datos de acuerdo a los parametros de desencriptacion.

Paralelizar no es mas que llevar la secuencia de bits que se desplazan como una sola fila, y transformarla en una columna. De esta forma si se tiene una secuencia serial de n bits, al paralelizar, el resultado es una columna de bits de n filas.

Esta accion de paralelizar la llevará acabo el circuito integrado 74HC595 tambien conocido como Registro de desplazamiento. Un chip con 3 entradas y 8 salidas digitales.

\begin{figure}[!ht]
\includegraphics[width=5cm]{74HC595.jpg}
\centering
\caption{Pines IC 74HC595}
\end{figure}

\noindent\textbf{Salidas:}\\ 
\indent Q0-Q7 (pines 1-7 y 15 )\\
\textbf{Entradas:}\\
\indent DS: entrada de datos (pin 14)\\
\indent STCP: entrada de la señal de reloj (pin 12)\\
\indent SHCP: entrada del pulso para liberar los datos (pin 11)


\section{Analisis del problema} \label{analisis}



\section{Conclusiones} \label{conclusiones}


\bibliographystyle{IEEEtran}
\bibliography{references}

\end{document}
